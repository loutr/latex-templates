\documentclass{article}
\usepackage[utf8]{luainputenc}
\usepackage[T1]{fontenc}
\usepackage[french]{babel}
\usepackage[table]{xcolor}
\usepackage[top=2cm, bottom=1.5cm]{geometry}
\usepackage{fancyhdr}
\usepackage{hyperref}

\renewcommand{\headrulewidth}{0pt}

\title{Exploration sous-marine autonome \\ \normalsize Travaux interdisciplinaires personnels encadrés}
\author{Pierre-Louis \textsc{Lamaze}, Guillaume \textsc{Richaud} et Lucas \textsc{Tabary}}
\date{}

\begin{document}
\maketitle
\thispagestyle{fancy}
\rhead{Lycée Roosevelt, Reims -- MPSI 2018-2019}
\cfoot{}

\section*{Mots-clés et répartition du travail}

\begin{center}
  \rowcolors{2}{gray!25}{white}
  \begin{tabular}{ll}
    \rowcolor{gray!50}
    \bf Mots-clés & \bf Keywords\\
    Génération procédurale & Procedural generation \\
    Fonction de bruit & Noise function \\
    Modélisation en 3D & 3D modelisation \\
    Topographie & Topography / HeightMap \\
    Géométrie différentielle & Differencial geometry \\
    Simulation informatique & Simulation \\
    Heuristiques & Heuristics \\
    Apprentissage approfondi & Deep learning \\
    Réseaux neuronaux & Neural networks \\
    Reconaissance d'espèce & Species recognition \\
  \end{tabular}
\end{center}

\paragraph{} Cette étude sera envisagée sous différents angles~:
\begin{itemize}
  \item Modélisation et génération d'un environnement similaire aux fonds sous-marins à des fins de simulations \cite{genevaux:tel-01196438, guerin:tel-01635126, McHugh:2016, gustavson:2005} (Pierre-Louis \textsc{Lamaze})~;
  \item Aperçu du dilemme exploitation/exploration dans le déplacement en milieu marin. \cite{sigaud:inria-00326864, benzaki:2017, manhaes:2017} (Guillaume \textsc{Richaud})~;
  \item Utilisation d'un algorithme d'apprentissage approfondi pour repérer et classer des espèces vivant dans les fonds-marins \cite{3B1B:2018, Branson:2014, Rathi:2018} (Lucas \textsc{Tabary}).
\end{itemize}

\bibliographystyle{siam}
\bibliography{../references.bib}

\end{document}
