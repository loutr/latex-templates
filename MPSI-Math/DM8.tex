\documentclass{article}
\usepackage[utf8]{luainputenc}
\usepackage[T1]{fontenc}
\usepackage[french]{babel}
\DecimalMathComma
\usepackage{amssymb}
\usepackage{amsmath}
\usepackage{mathrsfs}
\usepackage[top=1.9cm, bottom=1.9cm, left=1.9cm, right=1.9cm]{geometry}

\title{\textbf{DM \No 8~: Irrationalité de $e$}}
\author{Lucas \textsc{Tabary}}
\date{}
% \underset pour texte sous un symbole
\begin{document}
  \maketitle
  \hrulefill

  \vspace*{2cm}
  \hrulefill

  \section*{Irrationnalité de la limite}

  Soient les suites $\left(u_n\right)_{n>0}$ et $\left(v_n\right)_{n>0}$ définies par~:
  \begin{displaymath}
    u_n = \sum_{k=0}^n \frac{1}{k!} \quad\text{et}\quad v_n = u_n + \frac{1}{n\times n!}
  \end{displaymath}
  Montrons que ces suites sont adjacentes pour prouver l'existence de leur limite. Déterminons donc~:
  \begin{align*}
    u_{n+1} - u_n &= \sum_{k=0}^{n+1} \frac{1}{k!} - \sum_{k=0}^n \frac{1}{k!} = \frac{1}{(n + 1)!} > 0 \Longrightarrow \text{$u_n$ est croissante} \\
    v_{n + 1} - v_n &= u_{n + 1} + \frac{1}{(n + 1)(n + 1)!} - \left(u_n + \frac{1}{n\times n!}\right) =
    \frac{1}{(n + 1)!} + \frac{1}{(n + 1)(n + 1)!} - \frac{\frac{n + 1}{n}\times (n + 1)}{(n + 1)(n + 1)!} \\
    &= \frac{n + 2 - \frac{(n + 1)^2}{n}}{(n + 1)(n + 1)!} = - \frac{1}{n(n + 1)(n + 1)!} < 0 \Longrightarrow \text{$v_n$ est décroissante} \\
    u_n - v_n &= -\frac{1}{n\times n!} \underset{n\to\infty}{\longrightarrow} 0  \quad\text{par limite de quotient}
  \end{align*}
  On en conclut que $u$ et $v$ sont adjacentes et ont pour limite commune $\lambda\in\mathbb{R}$. De plus on a~:
  \begin{equation}\label{ADJACENTE}
    \forall n\in\mathbb{N}^*,\ u_n < \lambda < v_n
  \end{equation}

  Il est possible d'obtenir une première approximation avec une erreur $\varepsilon$ de $\lambda$ en déterminant le rang $n$ tel que l'intervalle $[u_n,\ v_n]$, de longueur $\frac{1}{n\times n!}$, soit de longueur $\varepsilon$. Pour $\varepsilon=10^{-4}$, on a, pour $n = 7$~:
  \begin{displaymath}
    \frac{1}{7 \times 7!} \approx 2,83\times 10^{-5} < 10^{-4} \Longrightarrow \vert u_7 - \lambda \vert < 10^{-4}
  \end{displaymath}

  On considère maintenant l'inéquation (\ref{ADJACENTE}) au rang $n + 3$, en considérant de plus la décroissance de $v$. D'où~:
  \begin{equation}\label{APPROX}
    u_{n + 3} \leqslant \lambda \leqslant v_{n + 3} \leqslant v_{n + 1} \Longrightarrow -v_{n + 1} \leqslant -\lambda \leqslant -u_{n + 3} \Longrightarrow v_n - v_{n + 1} \leqslant v_n - \lambda \leqslant v_n - u_{n + 3}
  \end{equation}
  À partir de cette équation, on obtient $\forall n\in\mathbb{N}^*,\ n^3n!(v_n - v_{n + 1}) \leqslant n^3n!(v_n - \lambda) \leqslant n^3n!(v_n - u_{n + 3})$, développons les membres de gauche et de droite de cette inégalité.
  \begin{align*}
    n^3n!(v_n - v_{n + 1}) &= \frac{n^3n!}{n(n + 1)(n + 1)!} = \frac{n^2}{(n+1)^2} = \frac{n^2}{n^2 + 2n + 1} \underset{\infty}{\sim} \frac{n^2}{n^2} \underset{n\to\infty}{\longrightarrow} 1
  \end{align*}
  \begin{align*}
    n^3n!(v_n - u_{n + 3}) &= n^3n!\left(\frac{1}{n\times n!} + \sum_{k = 0}^n \frac{1}{k!} - \sum_{k = 0}^{n+3} \frac{1}{k!}\right) = n^3n!\left(\frac{1}{n\times n!} - \frac{1}{(n+1)!} - \frac{1}{(n+2)!} - \frac{1}{(n+3)!}\right) \\
    &= n^3n!\left(\frac{(n+1)(n+2)(n+3)}{n(n+3)!}-\frac{n(n+2)(n+3)}{n(n+3)!} - \frac{n(n+3)}{n(n+3)!} - \frac{n}{n(n+3)!}\right) \\
    &= n^3n!\left(\frac{(n+2)(n+3)[(n+1)-n]-n(n+3)-n}{n(n+3)!}\right) = n^3n!\left(\frac{(n+3)[(n+2)-n] - n}{n(n+3)!}\right) \\
    &= n^3n! \times \frac{n+6}{n(n+3)!} = \frac{n^2(n+6)}{(n+1)(n+2)(n+3)} \underset{\infty}{\sim} \frac{n^3}{n^3} \underset{n\to\infty}{\longrightarrow} 1
  \end{align*}

  En appliquant le théorème des gendarmes avec l'inéquation (\ref{APPROX}), on obtient~:
  \begin{displaymath}
    \lim_{n\to\infty} n^3n!(v_n - \lambda) = 1 \iff v_n - \lambda \underset{\infty}{\sim} \frac{1}{n^3n!} \iff
    v_n - \lambda = \frac{1}{n^3n!} + o\left(\frac{1}{n^3n!}\right)
  \end{displaymath}
  On en conclut qu'approximativement $|v_n - \lambda| = \frac{1}{n^3n!}$, or pour $n = 5$, $\frac{1}{5\times 5!} \approx 6,66 \times 10^{-5} < 10^{-4}$. On en déduit que le terme $v_5$ est égale à la limite avec une erreur inférieure à $\varepsilon = 10^{-4}$.

  \paragraph{}
  On pose maintenant, $\forall n\in\mathbb{N}^*,\ a_n = n\cdot n!\cdot u_n$. Montrons par récurrence que $a_n$ est un nombre entier. On a donc $H_n: a_n \in\mathbb{N}$. À $n = 1$, $a_1 = 1 \cdot 1! \times u_1 = 1 \in\mathbb{N}$. On suppose maintenant que pour $n$, $H_n$ est vraie. On a~:
  \begin{displaymath}
    a_{n+1} = (n+1)(n+1)! \cdot u_{n+1} = (n+1)(n+1)!\left[u_n + \frac{1}{(n+1)!}\right] = (n+1)\left[(n+1)n!\cdot u_n  + 1\right] = (n+1)((n+1)a_n + 1)
  \end{displaymath}
  Or $a_n \in\mathbb{N}$, la somme et la multiplication étant stables dans $\mathbb{N}$, on en déduit $a_{n+1} \in\mathbb{N}$. Donc $H_{n+1}$ est vraie, est d'après le principe de raisonnement par récurrence, $\forall n\geqslant 1,\ a_n \in\mathbb{N}$.

  \paragraph{}
  Reprenons l'équation (\ref{ADJACENTE}), on a~:
  \begin{equation}\label{ABSURDE}
    \forall n\geqslant 1,\ u_n < \lambda < v_n \iff \frac{n\cdot n!}{n\cdot n!}u_n < \lambda < \frac{n\cdot n!}{n\cdot n!}\left(u_n + \frac{1}{n\cdot n!}\right)
    \iff \frac{a_n}{n\cdot n!} < \lambda < \frac{n\cdot n! \cdot u_n + 1}{n\cdot n!} = \frac{a_n + 1}{n\cdot n!}
  \end{equation}
  Montrons par l'absurde que $\lambda$ est irrationnel. Supposons donc que~: $\lambda\in\mathbb{Q}\Rightarrow \exists (p,\ q)\in\mathbb{N}\times\mathbb{N}^*,\ q \neq 0,\ \lambda = \frac{p}{q}$. Injectons l'expression de $\lambda$ dans l'inéquation (\ref{ABSURDE}). Celle-ci est vraie pour toute valeur de $n$ positive, en particulier pour $n = q$. On a donc~:
  \begin{displaymath}
    \frac{a_q}{q\cdot q!} < \frac{p}{q} < \frac{a_q + 1}{q\cdot q!} \Longrightarrow a_q < p\cdot q! < a_q + 1 \Longrightarrow p\cdot q! \in ]\!] a_q,\ a_q + 1 [\![ = \varnothing
  \end{displaymath}
  Ce qui est absurde. On en déduit que l'hypothèse initiale est fausse, par conséquent $\boxed{\lambda\not\in\mathbb{Q}}$.

  \section*{Valeur de la limite}
  On pose~:
  \begin{displaymath}
    \forall n \in\mathbb{N}^*,\ \forall p \in [\![2,\ n]\!],\ \mathcal{K}(n,\ p) = \prod_{k = 1}^{p - 1} \left(1 - \frac{k}{n}\right)
  \end{displaymath}
  Démontrons premièrement par récurrence sur $p\in[\![2,\ n]\!]$, $H_p: 0 \leqslant 1 - \mathcal{K}(n,\ p) \leqslant \frac{p(p - 1)}{2n}$. On prendra $n \in\mathbb{N}^*$. Pour $p = 2$~:
  \begin{displaymath}
    1 - \mathcal{K}(n,\ 1) = 1 - \prod_{k = 1}^{1} \left(1 - \frac{k}{n}\right) = 1 - 1 + \frac{1}{n} = \frac{1}{n} \geqslant 0 \quad\text{ et }\quad \frac{2(2-1)}{2n} = \frac{1}{n} \geqslant \frac{1}{n}
  \end{displaymath}
  $H_2$ est donc vraie. on suppose maintenant que pour $p \geqslant 2$, $H_p$ est vraie, d'où~:
  \begin{align*}
    0 \leqslant 1 - \mathcal{K}(n,\ p) \leqslant \frac{p(p - 1)}{2n} &\iff 0 \leqslant \left(1 - \frac{p}{n}\right)\left[1 - \mathcal{K}(n,\ p)\right] \leqslant \left(1 - \frac{p}{n}\right)\frac{p(p - 1)}{2n} \quad \because p \leqslant n \Rightarrow 0 < 1 - \frac{p}{n} < 1 \\
    &\iff 0 \leqslant 1 - \frac{p}{n} - \mathcal{K}(n,\ p + 1) \leqslant \left(1 - \frac{p}{n}\right)\frac{p(p - 1)}{2n} \leqslant \frac{p(p - 1)}{2n} \\
    &\iff 0 \leqslant \frac{p}{n} \leqslant 1 - \mathcal{K}(n,\ p + 1) \leqslant \frac{p}{n} + \frac{p^2 - p}{2n} = \frac{p^2 + p}{2n} = \frac{(p+1)p}{2n}
  \end{align*}
  $H_{p+1}$ est donc vraie. D'après le principe de raisonnement par récurrence, on obtient donc~:
  \begin{equation}\label{INEQK}
    \forall n>0,\ \forall p\in[\![2,\ n]\!],\ 0 \leqslant 1 - \mathcal{K}(n,\ p) \leqslant \frac{p(p - 1)}{2n}
  \end{equation}

  On pose maintenant $\forall n > 0,\ w_n = \left(1 + \frac{1}{n}\right)^n$. Étudions la différence suivante~:
  \begin{align*}
    u_n - w_n &= \sum_{k=0}^n \frac{1}{k!} - \left(1 + \frac{1}{n}\right)^n = \sum_{k=0}^n \frac{1}{k!} - \sum_{k=0}^n\binom{n}{k}\left(\frac{1}{n}\right)^k 1^{n-k} = \sum_{k=0}^n \frac{1}{k!} - \sum_{k=0}^n \frac{n!}{k!(n - k)!}\times \frac{1}{n^k} \\
    &= \sum_{k=0}^n \frac{1}{k!}\left[1 - \frac{n!}{(n - k)! n^k}\right] = \sum_{k=2}^n \frac{1}{k!}\left[1 - \frac{n!}{(n - k)! n^k}\right] + \underbrace{\frac{1}{0!}\left(1 - \frac{n!}{n!n^0}\right) + \frac{1}{1!}\left(1 - \frac{n!}{(n-1)!n^1}\right)}_{0}
  \end{align*}
  Étudions séparément une partie du calcul~:
  \begin{align*}
    \frac{n!}{(n - p)! n^p} &= \frac{(n-p)!(n-(p-1))\cdots(n-2)(n-1)n}{(n-p)!} \times \frac{1}{n^p} = \frac{1}{n^{p-1}}\times\prod_{k=1}^{p-1}(n-k) = \prod_{k=1}^{p-1} \frac{1}{n} \times \prod_{k=1}^{p-1}(n-k)  \\
    &= \prod_{k=1}^{p-1}\frac{1}{n}(n - k)=\prod_{k=1}^{p-1}\left(1 - \frac{k}{n}\right) = \mathcal{K}(n,\ p)
  \end{align*}
  En injectant dans la relation donnée précédemment, on a donc ($\mathcal{K}(n,\ p)$ existe car $p\in[\![2,\ n]\!]$)~:
  \begin{equation}\label{SOMMEK}
    u_n - w_n = \sum_{p=2}^{n}\frac{1 - \mathcal{K}(n,\ p)}{p!}
  \end{equation}

  Reprenons maintenant l'inégalité (\ref{INEQK}). On peut effectuer la somme de cette inégalité pour $p$ allant de $2$ à $n$ car l'inégalité est vraie pour toutes ces valeurs. On a~:

  \begin{align*}
    \forall n>0,\ \forall p\in[\![2,\ n]\!],\ 0 \leqslant 1 - \mathcal{K}(n,\ p) \leqslant \frac{p(p - 1)}{2n}
    \iff 0 \leqslant \frac{1 - \mathcal{K}(n,\ p)}{p!} \leqslant \frac{p(p - 1)}{p!\cdot 2n} = \frac{1}{2n(p-2)!} \\
    \Longrightarrow \sum_{p=2}^n 0 \leqslant \sum_{p=2}^n \frac{1 - \mathcal{K}(n,\ p)}{p!} \leqslant \sum_{p=2}^n \frac{1}{2n(p-2)!} =
    \frac{1}{2n}\sum_{i=0}^{n-2} \frac{1}{i!} = \frac{u_{n-2}}{2n} \text{ or d'après (\ref{ADJACENTE}) } \forall n\in\mathbb{N},\ u_n < \lambda \\
    \therefore 0 \leqslant u_n - w_n \leqslant \frac{\lambda}{2n} \Longrightarrow u_n - \frac{\lambda}{2n} \leqslant w_n \leqslant u_n
  \end{align*}

  Néanmoins $\underset{n\to\infty}{\lim}\ \dfrac{\lambda}{2n} = 0 \Rightarrow \underset{n\to\infty}{\lim}\ u_n - \dfrac{\lambda}{2n} = \lambda$. On applique donc le théorème des gendarmes dans l'inégalité déterminée précédemment. On en conclut que $(w_n)$ converge vers la même valeur $\lambda$ que $(u_n)$. Concluons~:
  \begin{align*}
    \ln\left(1 + \frac{1}{n}\right) \underset{\infty}{\sim} \frac{1}{n}\ \because \frac{1}{n}\underset{n\to\infty}{\longrightarrow} 0\quad \Longrightarrow n\ln\left(1 + \frac{1}{n}\right) \underset{\infty}{\sim} \frac{n}{n} = 1
    \Longrightarrow \lim_{n\to\infty} n\ln\left(1 + \frac{1}{n}\right) = 1 \\
    \therefore \lim_{n\to\infty} w_n = \lim_{n\to\infty} \exp\left[n\ln\left(1 + \frac{1}{n}\right)\right] = \lim_{N\to 1} \exp N = \boxed{e = \lambda}
  \end{align*}
\end{document}
