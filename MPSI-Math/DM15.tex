\documentclass{article}
\usepackage[utf8]{luainputenc}
\usepackage[T1]{fontenc}
\usepackage[french]{babel}
\usepackage{enumitem}

\DecimalMathComma

\newcommand{\prob}{\mathbb{P}}
\newcommand{\esp}{\mathbb{E}}

\usepackage[top=3cm, bottom=3cm, left=2.3cm, right=2.3cm]{geometry}
\usepackage{amssymb, amsmath}

\title{\textbf{DM \No 15~: Un problème de probabilités}}
\author{Lucas \textsc{Tabary}}
\date{}

\begin{document}
  \maketitle
  \hrulefill

  \vspace{1.2cm}
  \hrulefill

  \section*{Énoncé}
  On effectue une série de tirages avec remises dans une urne contenant $n$ boules numérotées de 1 à $n$. On note $X_k$ la variable aléatoire égale à la valeur de la boule tirée au $k$-ème tirage ($k > 0$). On note alors $S_k$ la variable aléatoire comptabilisant la somme des valeurs obtenues jusqu'au $k$-ème tirage inclus. On a~:
  \[
    S_k = \sum_{i=1}^k X_i
  \]
  On considère enfin la variable aléatoire $T_n$ égale au nombre de tirages nécessaires pour que, pour la première fois, la somme des boules obtenues soit supérieure ou égale à $n$. L'étude se fera dans un espace probabilisé $(\Omega,\, \mathcal{P}(\Omega),\, \prob)$.

  \section*{Partie A}
  \begin{enumerate}
    \item Déterminons la loi de $X_k$. On remarque préalablement que tous les tirages sont réalisés de façon identique et indépendante. On peut dès lors considérer que chaque valeur a la même probabilité d'être désignée parmi $n$ boules. Donc~:
    \[
      \forall k \in [\![1,\, n]\!],\; X_k \hookrightarrow \mathcal{U}_{[\![1,\, n]\!]} \Longrightarrow \prob(X_k = k) = \frac{1}{n} \text{ et } \esp(X_k) = \frac{n + 1}{2}
    \]

    \item \begin{enumerate}
      \item \label{Q2} Déterminons la valeur de $T_n(\Omega)$. Dans le meilleur cas, on obtient la boule $n$ au premier tirage, et alors $T_n = 1$. Dans le pire des cas on progresse au rythme de $1$ par tirage, on obtiendra donc $n$ au bout de $n$ tirages, d'où $T_n = n$. Pour obtenir un nombre entre ces deux bornes, qu'on notera $k \in ]\!]1,\, n[\![$, on peut tirer $k - 1$ fois la boule $1$ puis une fois la boule $n - k + 1 \in [\![1,\, n]\!]$, ce qui donnera bien une valeur totale de $n$ pour $k$ tirages~: toutes les valeurs sont atteignables et on a~: $T_n(\Omega) = [\![1,\, n]\!]$.

      \item Le seul moyen d'obtenir $n$ en 1 coup est de tirer la boule $n$, au premier coup \textit{(sic)}. Soit, par équiprobabilité~: $\prob(T_n = 1) = \frac{1}{n}$.

      \item On réalise $n$ tirages pour obtenir une valeur totale supérieure ou égale à $n$. Au $n-1$-ème tirage, si une valeur est supérieure à 1, on a alors
      \[
        \sum_{k=1}^{n-1} X_k > \sum_{k=1}^{n-1} 1 = n - 1 \Longrightarrow \sum_{k=1}^{n-1} X_k \geqslant n
      \]
      Ce qui est absurde car on aurait alors $T_n = n - 1$. Le dernier n'a pas d'importance car dans tous les cas on obtient une valeur supérieure ou égale à $n$. D'où~:
      \[
        \prob(T_n = n) = \left(\frac{1}{n}\right)^{n-1} \cdot 1
      \]
    \end{enumerate}
    \item Pour $T_2$, on exploite directement les résultats des questions précédentes~: on a alors~:
    \begin{itemize}
      \item $\prob(T_2 = 1) = \frac{1}{2}$;
      \item $\prob(T_2 = 2) = \left(\frac{1}{2}\right)^{2 - 1} = \frac{1}{2}$.
    \end{itemize}

    \item Pour $T_3$, on utilise encore les formules~:
    \begin{itemize}
      \item $\prob(T_3 = 1) = \frac{1}{3}$;
      \item $\prob(T_3 = 3) = \left(\frac{1}{3}\right)^{3 - 1} = \frac{1}{9}$~;
      \item En utilisant le système complet d'évènements associé à $T_3$,
      \[
        \prob(T_3 = 2) = 1 - [\prob(T_3 = 1) + \prob(T_3 = 3)] = 1 - \frac{4}{9} = \frac{5}{9}
      \]
    \end{itemize}
    D'où $\esp(T_3) = 1 \cdot \frac{1}{3} + 3 \cdot \frac{1}{9} + 2 \cdot \frac{5}{9} = \frac{2}{3} + \frac{10}{9} = \frac{16}{9}$.
  \end{enumerate}

  \section*{Partie B}

  \begin{enumerate}[resume]
    \item On reconsidère les pires et meilleurs cas décrit dans la question \ref{Q2}. On a alors~:
    \[
      k = \sum_{i=1}^k 1 \leqslant S_k = \sum_{i=1}^k X_i \leqslant \sum_{i=1}^k n = kn
    \]
    Toutes les valeurs étant atteignables, on a~: $S_k(\Omega) = [\![k;\, kn]\!]$.

    \item \begin{enumerate}
      \item Par définition de $S_k$ sur l'intervalle considéré~:
      \[
        S_{k+1} = \sum_{i=1}^{k+1} X_i = X_{k+1} + \sum_{i=1}^k X_i = X_{k+1} + S_k
      \]

      \item \label{Q6B} On considère le système complet d'évènements associé à $S_k$, soit $\{(S_k = j),\, j \in [\![k;\, kn]\!]\}$, on remarque que $X_{k+1} = S_{k+1} - S_k$, on utilise alors la loi des probabilités totales~:
      \[
        \prob(S_{k+1} = i) = \sum_{j=k}^{kn} \underbrace{\prob_{(S_k = j)}(S_{k+1} = i)}_{\prob(X_{k+1} = i - j)}\prob(S_k = j)
      \]
      Or $\prob(X_{k+1} = i - j) =
      \left\{
      \begin{aligned}
        \frac{1}{n} & \text{ si } i-j\in [\![1,\, n]\!] \\
        0           & \text{ sinon}
      \end{aligned}
      \right.$. Les cas non nuls se présentent pour $1 - n \leqslant j \leqslant i - 1$. On remplace donc dans l'expression précédente et alors~:
      \[
        \prob(S_{k+1} = i) = \sum_{j=k}^{i - 1} \underbrace{\prob(X_{k+1} = i - j)}_{\frac{1}{n}}\prob(S_k = j) + \sum_{j=i}^{kn} \underbrace{\prob(X_{k+1} = i - j)}_{0}\prob(S_k = j)
        = \frac{1}{n} \sum_{j=k}^{i - 1} \prob(S_k = j)
      \]
    \end{enumerate}

    \item \begin{enumerate}
      \item On rappelle directement la relation de Pascal, pour $0 < k < n$~:
      \[
        {j \choose k} = {j - 1 \choose k - 1} + {j - 1 \choose k} \Longrightarrow {j - 1 \choose k - 1} =   {j \choose k} - {j - 1 \choose k}
      \]

      \item On somme les deux termes de l'égalité précédente de $j = k$ à $i - 1$ (avec nécessairement $i - 1 \geqslant k$)~:
      \[
        \sum_{j=k}^{i - 1} {j - 1 \choose k - 1} = \sum_{j=k}^{i - 1} {j \choose k} - \sum_{j=k}^{i - 1} {j - 1 \choose k} = \sum_{j=k + 1}^{i} {j - 1 \choose k} - \sum_{j=k}^{i - 1} {j - 1 \choose k} = {i - 1 \choose k} -
        \underbrace{k - 1 \choose k}_{0}
      \]
      On obtient alors la formule demandée.

      \item On considère $\mathcal{H}_k$, qu'on se propose de démontrer par récurrence forte, pour $k$ un entier dans $[\![1,\, n]\!]$.
      \[
        \mathcal{H}_k \colon \forall i \in [\![k,\, kn]\!],\ \prob(S_k = i) = \frac{1}{n^k}{i - 1 \choose k - 1}
      \]
      À $n = 1$, on a bien, par équiprobabilité~: $\forall i \in [\![1,\, n]\!],\, \prob(S_1 = i) = \prob(X_1 = i) = \frac{1}{n} = \frac{1}{n} {i - 1 \choose i - 1}$. La propriété est initialisée. On suppose maintenant que $\mathcal{H}_k$ est vérifiée, pour $k \in [\![1,\, n-1]\!]$. On a alors, d'après la conclusion de la question \ref{Q6B}~:
      \[
       \forall i \in [\![k,\, n]\!],\ \prob(S_{k+1} = i) = \frac{1}{n} \sum_{j=k}^{i - 1} \prob(S_k = j) = \frac{1}{n} \sum_{j=k}^{i - 1} \frac{1}{n^k}{j - 1 \choose k - 1} = \frac{1}{n^{k+1}} \sum_{j=k}^{i - 1} {j - 1 \choose k - 1}
      = \frac{1}{n^{k+1}}{j \choose k - 1}
      \]
      On a donc $\mathcal{H}_k \Rightarrow \mathcal{H}_{k+1}$. On conclut à partir du principe de raisonnement par récurrence~:
      \[
        \forall k \in [\![1,\, n]\!],\ \forall i \in [\![k,\, kn]\!],\ \prob(S_k = i) = \frac{1}{n^k}{i - 1 \choose k - 1}
      \]
    \end{enumerate}

    \item \begin{enumerate}
      \item À partir de l'énoncé, on a $(T_n = k) = (S_k \geqslant n) \cap (S_{k-1} < n)$. De là, si $T_n > k$, alors $T_n \neq k$ et par négation $S_k < n$ car sinon, $S_{k-1} \geqslant n$, et alors $T_n \leqslant k - 1$, ce qui est absurde. Par ailleurs, si $S_k < n$, alors par définition $T_n > k$. On établit finalement~:
      \[
        (T_n > k) = (S_k < n) \Longrightarrow \prob(T_n = k) = \prob(S_k < n)
      \]

      \item \label{QF} Par l'égalité précédente~:
      \[
        \prob(T_n > k) = \prob(S_k < n) = \sum_{j=k}^{n-1} \prob(S_k = j) = \sum_{j=k}^{n-1} \frac{1}{n^k} {j-1 \choose k - 1} = \frac{1}{n^k} \sum_{j=k}^{n-1} {j-1 \choose k - 1} = \frac{1}{n^k} {n-1 \choose k}
      \]


      % \begin{align*}
      %   (T_n > k) = \bigcup_{k<j\leqslant n} (T_n = j) \Longrightarrow \prob(T_n > k) &= \sum_{k<j\leqslant n} (T_n = j) = \sum_{j=k+1}^n \prob_{(T_n = j)}(S_k < n) \prob(T_n = j) \\
      %   &= \sum_{j=1}^n \prob_{(T_n = j)}(S_k < n) \prob(T_n = j) - \sum_{j=1}^k \underbrace{\prob_{(T_n = j)}_{0}(S_k < n)} \prob(T_n = j) \\
      %   &= \prob(S_k < n) = \sum_{}
      % \end{align*}
    \end{enumerate}

    \item On utilise la définition de l'espérance dans laquelle on fait apparaître une somme triangulaire~:
    \[
      \esp(T_n) = \sum_{k=1}^n k \prob(T_n = k) = \sum_{k = 1}^n \sum_{j=1}^k \prob(T_n = k) = \sum_{j=1}^n \underbrace{\sum_{k=j}^n \prob(T_n = k)}_{\prob(T_n \geqslant j)} = \sum_{j=1}^n \prob(T_n > j - 1) = \sum_{k=0}^{n-1} \prob(T_n > k)
    \]
    On peut alors injecter l'expression obtenue en \ref{QF} et on obtient~:
    \[
      \esp(T_n) = \sum_{k=0}^{n-1} \frac{1}{n^k}{n-1 \choose k} = \sum_{k=0}^{n-1} {n-1 \choose k} \left(\frac{1}{n}\right)^k \cdot 1^{n-1-k} = \left(\frac{1}{n} + 1\right)^{n-1}
    \]
    en utilisant le binôme de Newton.

    \item Étudions maintenant la limite de cette expression. On utilisera des équivalents.
    \begin{align*}
      (n-1)\ln\left(1 + \frac{1}{n}\right) \underset{\infty}{\sim} n \cdot \frac{1}{n} \longrightarrow 1 \\
      \Longrightarrow \left(1 + \frac{1}{n}\right)^{n-1} =
      \exp\left[(n-1)\ln\left(1 + \frac{1}{n}\right)\right] \underset{n\to\infty}{\longrightarrow} \exp 1 \\
      \therefore \lim_{n\to\infty} \esp(T_n) = e
    \end{align*}
  \end{enumerate}
\end{document}
