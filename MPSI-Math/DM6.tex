\documentclass{article}
\usepackage[utf8]{luainputenc}
\usepackage[T1]{fontenc}
\usepackage[french]{babel}
\DecimalMathComma
\usepackage{amssymb}
\usepackage{amsmath}
\usepackage{mathrsfs}
\usepackage[top=1.9cm, bottom=1.9cm, left=1.9cm, right=1.9cm]{geometry}
\usepackage{tikz,tkz-tab}

\newcommand{\oo}[1]{+ o\!\left(#1\right)}

\title{\textbf{DM \No 6~: Développement limité}}
\author{Lucas \textsc{Tabary}}
\date{}

\begin{document}
  \maketitle
  \hrulefill

  \vspace*{2cm}
  \hrulefill

    \section{Calcul du $DL_5(0)$ de $\tan$}
    \subsection{Par intégrations successives}

    \paragraph{a.} La fonction $\tan$ est définie au voisinage de 0 (sur $]-\pi/2,\,\pi/2[$) et y est dérivable. On a $\tan' = 1 + \tan^2 = \frac{1}{\cos^2}$.

    \paragraph{b.} Puisque $\tan$ est dérivable en 0, elle admet un $DL_1(0)$~:
    \begin{displaymath}
      \tan x \underset{0}{=} \tan(0) + x\tan'(0) + o(x) \underset{0}{=} 0 + x(1 + 0^2) + o(x) \underset{0}{=} x + o(x)
    \end{displaymath}

    \paragraph{c.} La fonction $\tan$ étant impaire, les coefficients associés aux indices pairs sont nuls. On en déduit~:
    \begin{displaymath}
      \tan x \underset{0}{=} x + o(x) \underset{0}{=} x \oo{x^2}
    \end{displaymath}

    \paragraph{d.} On a bien $1 + x^2 = 1 + x^2 \oo{x^2}$. On compose donc avec le développement limité de $\tan$ puisque~:
    \begin{displaymath}
      \lim_{x\to 0}\tan x = 0 \Rightarrow 1 + \tan^2(x) \underset{0}{=} 1 + (x)^2 \oo{x^2}
    \end{displaymath}

    \paragraph{e.} Par intégration, on a donc~:
    \begin{displaymath}
      \tan x = \int(1 + \tan^2 x) \underset{0}{=} x + \frac{1}{3}x^3 \oo{x^3}
    \end{displaymath}

    \paragraph{f.} On compose à nouveau avec le développement à l'ordre 4 de $1 + x^2 = 1 + x^2 \oo{x^4}$. Car les $DL_3(0)$ et $DL_4(0)$ de $\tan$ sont identiques, on compose comme précédemment avec $1 + x^2$. On a donc~:
    \begin{displaymath}
      \lim_{x\to 0}\tan x = 0 \Rightarrow 1 + \tan^2(x) \underset{0}{=} 1 + \left(x + \frac{1}{3}x^3\right)^2 \oo{x^4} \underset{0}{=} 1 + x^2 + \frac{2}{3}x^4 \oo{x^4}
    \end{displaymath}

    On réalise donc une deuxième intégration pour obtenir le $DL_5(0)$ de $\tan$~:
    \begin{displaymath}
      \tan x = \int(1 + \tan^2 x) \underset{0}{=} x + \frac{1}{3}x^3 + \frac{2}{15}x^5 \oo{x^5}
    \end{displaymath}

    \subsection{À l'aide de la bijection réciproque}

    \paragraph{a.} On a $\lim_{x\to 0} x^2 = 0$, on peut donc composer avec le $DL_4(0)$ (connu) de $\frac{1}{1+x} = 1 - x + x^2 - x^3 + x^4 \oo{x^4}$. On a~:
    \begin{displaymath}
      \frac{1}{1+x^2} \underset{0}{=} 1 - (x^2) + (x^2)^2 - (x^2)^3 + (x^2)^4 \oo{x^4} \underset{0}{=} 1 - x^2 + x^4 \oo{x^4}
    \end{displaymath}
    Par intégration, on en déduit~:
    \begin{displaymath}
      \arctan x = \int\frac{1}{1 + x^2} \underset{0}{=} x - \frac{1}{3}x^3 + \frac{1}{5}x^5 \oo{x^5}
    \end{displaymath}

    \paragraph{b.} Comme évoqué dans la première partie, tan est impaire, donc en écrivant~:
    \begin{displaymath}
      \tan x \underset{0}{=} a_0 + a_1x + a_2x^2 + \dotsb + a_5x^5 \oo{x^5}
    \end{displaymath}
    On a $a_0 = a_2 = a_4 = 0$ car les coefficients associés aux puissances paires sont nuls. De plus $\lim_{x\to 0} \arctan x = 0$, on peut donc déterminer le $DL_5(0)$ associé à $\tan\circ\arctan$. On écrit~:
    \begin{align*}
      \tan\circ\arctan x &\underset{0}{=} a_1\left(x - \frac{1}{3}x^3 + \frac{1}{5}x^5\right) + a_3\left(x - \frac{1}{3}x^3 + \frac{1}{5}x^5\right)^3 + a_5\left(x - \frac{1}{3}x^3 + \frac{1}{5}x^5\right)^5 \oo{x^5} \\
      &\underset{0}{=} a_1x - \frac{a_1}{3}x^3 + \frac{a_1}{5}x^5 + a_3\left(x^2 - \frac{2}{3}x^4\right)\left(x - \frac{1}{3}x^3 + \frac{1}{5}x^5\right) + a_5x^5 +\oo{x^5} \\
      &\underset{0}{=} a_1x - \frac{a_1}{3}x^3 + \frac{a_1}{5}x^5 + a_3x^3 - \frac{a_3}{3}x^5 - \frac{2a_3}{3}x^5 + a_5x^5 \oo{x^5} \\
      &\underset{0}{=} a_1x + \left(a_3 - \frac{a_1}{3}\right)x^3 + \left(\frac{a_1}{5} - a_3 + a_5\right)x^5 \oo{x^5}
    \end{align*}

    Or par définition $\tan\circ\arctan x = x \underset{0}{=} x \oo{x^5}$, par unicité de la partie régulière du développement limité, on trouve en identifiant~:
    \begin{align*}
      \left\{
        \begin{array}{rl}
          a_1  &= 1 \\
          a_3 - \frac{a_1}{3} &= 0 \\
          \frac{a_1}{5} - a_3 + a_5 &= 0
        \end{array}
      \right.
      \iff\left\{
        \begin{array}{rl}
          a_1  &= 1 \\
          a_3 &= \frac{1}{3} \\
          a_5 &= a_3 - \frac{a_1}{5} = \frac{1}{3} - \frac{1}{5} = \frac{2}{15}
        \end{array}
      \right.
    \end{align*}
    On conclut en injectant dans la première expression du $DL_5(0)$ de $\tan$~:
    \begin{displaymath}
      \tan x \underset{0}{=} x + \frac{1}{3}x^3 + \frac{2}{15}x^5 \oo{x^5}
    \end{displaymath}

    \section{Une étude de fonction}
    \subsection{Étude de la tangente en 0}

    On considère la fonction $f\colon x \mapsto\left(\cosh x\right)^{1/x}$. On note $D$ son ensemble de définition. On a $\forall x\in\mathbb{R},\, \cosh x > 0$. La fonction est donc définie si, et seulement si $\frac{1}{x}$ est définie. Donc $D = \mathbb{R}^*$.

    On rappelle le $DL_3(0)$ de la fonction $\cosh$~:
    \begin{displaymath}
      \cosh x \underset{0}{=} 1 + \frac{1}{2}x^2 \oo{x^3} \Rightarrow \cosh(x) - 1 \underset{0}{=} \frac{x^2}{2} \oo{x^3}
    \end{displaymath}
    On connaît de plus le $DL_3(0)$ de la fonction $h\mapsto\ln(1+h)$, or $\lim_{x\to 0} \cosh(x) - 1 = 0$, on peut donc composer~:
    \begin{align*}
      \ln(1+h) \underset{0}{=} h - \frac{h^2}{2} + \frac{h^3}{3} \oo{x^3} \Rightarrow \ln(1 + \cosh(x) - 1) \underset{0}{=} \frac{x^2}{2} - \frac{1}{2}\left(\frac{x^2}{2}\right)^2 + \left(\frac{x^2}{2}\right)^3 \oo{x^3} \\
      \therefore\ \ln(\cosh x) \underset{0}{=} \frac{x^2}{2} \oo{x^3} \Rightarrow \frac{1}{x}\ln(\cosh x) \underset{0}{=} \frac{x}{2} \oo{x^2}
    \end{align*}
    De cette expression on déduit que $\lim_{x\to 0} \frac{1}{x}\ln(\cosh x) = 0$ car la limite d'une fonction en un point (si elle existe) correspond au terme constant de son développement limité au voisinage de ce point. On peut donc encore composer par la fonction $\exp$, sachant que~:
    \begin{align*}
      \exp x \underset{0}{=} 1 + x + \frac{x^2}{2} + \frac{x^3}{6} \oo{x^3}
      \Rightarrow \exp\left[\frac{1}{x}\ln(\cosh x)\right] \underset{0}{=} 1 + \frac{x}{2} + \frac{1}{2}\left(\frac{x}{2}\right)^2 + \frac{1}{6}\left(\frac{x}{2}\right)^3 \oo{x^3} \\
      \therefore f(x) = \left(\cosh x\right)^{1/x} \underset{0}{=} 1 + \frac{x}{2} + \frac{x^2}{8} \oo{x^2}
    \end{align*}

    On considère maintenant le prolongement par coninuité de $f$ sur $D' = \mathbb{R}$, qu'on notera aussi $f$. On détermine en effet la limite de $f$ en 0 (d'après la propriété énoncée précédemment) et on définit ainsi son image.
    \begin{displaymath}
      \lim_{x\to 0} f(x) = 1 = f(0)
    \end{displaymath}

    On détermine l'écriture de l'équation de la tangente en 0 notée $\mathscr{T}_0$ de $\mathscr{C}$ la courbe représentative de $f$ à partir de son $DL_1(0)$, obtenu par troncature du $DL_2(0)$. On a donc~:
    \begin{displaymath}
      \mathscr{T}_0\colon y = g(x) \text{ avec } g(x) = 1 + \frac{x}{2}
    \end{displaymath}
    On en déduit ainsi la position relative de la courbe par rapport à sa tangente à partir de l'écriture du développement limité de $f-g$ en 0~:
    \begin{displaymath}
      f(x) - g(x) \underset{0}{=} \frac{x^2}{8} \oo{x^2}
    \end{displaymath}
    Une fonction ayant localement le même signe que son développement limité, et sachant que $\forall x\in\mathbb{R}^*,\ \frac{x^2}{8} > 0$~; on en déduit que $\mathscr{C}$ est localement au-dessus de $\mathscr{T}_0$.

    \subsection{Étude des variations}

    Déterminons premièrement les limites en $\pm\infty$ de $f$. On réécrit la fonction $\cosh$~:
    \begin{displaymath}
      \cosh x = \frac{e^x + e^{-x}}{2} = e^x \frac{1 + e^{-2x}}{2} \Rightarrow \ln(\cosh x) = x + \ln\frac{1 + e^{-2x}}{2}
    \end{displaymath}
    On en déduit donc, par limite de produit, de fonctions composées~:
    \begin{align*}
      \lim_{x\to +\infty} \frac{1 + e^{-2x}}{2} = \frac{1}{2} \Rightarrow \lim_{x\to +\infty} \ln\frac{1 + e^{-2x}}{2} = -\ln 2 \Rightarrow \lim_{x\to +\infty} \frac{\ln\frac{1 + e^{-2x}}{2}}{x} = 0 \\
      \therefore \ln\frac{1 + e^{-2x}}{2} \underset{+\infty}{=} o(x) \text{ d'où } x \underset{+\infty}{\sim} x
      \Rightarrow x + \ln\frac{1 + e^{-2x}}{2} \underset{+\infty}{\sim} x
      \Rightarrow \ln(\cosh x) \underset{+\infty}{\sim} x
      \Rightarrow \frac{1}{x}\ln(\cosh x) \underset{+\infty}{\sim} 1
    \end{align*}
    Ce qui permet de finalement déterminer~:
    \begin{displaymath}
      \lim_{x\to +\infty} f(x) = \lim_{x\to +\infty} \exp\left[\frac{1}{x}\ln(\cosh x)\right] = \exp 1 = e
    \end{displaymath}

    On effectue maintenant un changement de variable de la dernière équivalence avec $y = -x$, sachant que $\lim_{x\to -\infty} -x = +\infty$, puis on utilise la parité de $\cosh$~:
    \begin{displaymath}
      \frac{1}{x}\ln(\cosh x) \underset{+\infty}{\sim} 1
      \Rightarrow \frac{1}{-x}\ln(\cosh -x) \underset{-\infty}{\sim} 1
      \Rightarrow \frac{1}{x}\ln(\cosh x) \underset{-\infty}{\sim} -1
    \end{displaymath}
    On peut donc déterminer~:
    \begin{displaymath}
      \lim_{x\to -\infty} f(x) = \lim_{x\to -\infty} \exp\left[\frac{1}{x}\ln(\cosh x)\right] = \exp (-1) = e^{-1}
    \end{displaymath}

    Étudions maintenant la dérivabilité de $f$ sur $D$. Elle y est dérivable par composition. On note $f'$ sa fonction dérivée associée et, $\forall x\in D,$
    \begin{align*}
      f'(x) = \left(\frac{1}{x}\ln(\cosh x)\right)'\exp\left[\frac{1}{x}\ln(\cosh x)\right] = \left(-\frac{1}{x^2}\ln(\cosh x) + \frac{1}{x}\frac{\sinh x}{\cosh x}\right)f(x) = \frac{1}{x^2}(x\tanh(x) - \ln(\cosh x))f(x)
    \end{align*}
    On définit donc sur $D$, $\varphi\colon x\mapsto x\tanh(x) - \ln(\cosh x)$, et on a $f'(x) = \frac{\varphi(x)}{x^2}f(x)$. $\varphi$ est dérivable sur $D$ par somme et composée. On a donc~:
    \begin{displaymath}
      \varphi'(x) = \tanh x + \frac{x}{\cosh^2 x} - \tanh x = \frac{x}{\cosh^2 x} > 0 \iff x > 0 \quad\because\forall x\in D,\ \cosh^2 x > 0
    \end{displaymath}
    De plus $\lim_{x\to 0} \varphi(x) = \varphi(0) = 0 \times \tanh 0 - \ln(\cosh 0) = 0$. On établit le tableau de variations de $\varphi$ à la figure \ref{varphi}, ainsi que son signe. On a donc $\forall x\in D,\ \varphi(x) > 0$.

    \begin{figure}[ht]
     \begin{center}
      \begin{tikzpicture}
       \tkzTabInit{$x$ / 1, $\varphi'(x)$ / 1, $\varphi$ / 1.5, $\varphi(x)$ / 1}{$-\infty$, 0, $+\infty$}
       \tkzTabLine{,-, d, +,}
       \tkzTabVar{+/, -C/0, +/}
       \tkzTabLine{,+,d,+,}
      \end{tikzpicture}
     \end{center}
     \caption{Tableau de variations de $\varphi$.}
     \label{varphi}
    \end{figure}


    On peut maintenant déterminer le signe de $f$ en considérant l'expression trouvée précédemment comme une équation différentielle linéaire homogène, car $x\mapsto\frac{\varphi(x)}{x^2}$ est continue par quotient~:
    \begin{displaymath}
      f'(x) - \frac{\varphi(x)}{x^2}f(x) = 0 \Rightarrow f(x) = \lambda\exp\int\frac{\varphi(x)}{x^2},\ \lambda\in\mathbb{R}
    \end{displaymath}

    De plus, d'après l'étude de $\varphi$, $\forall x\in D,\ \frac{\varphi(x)}{x^2} > 0 \Rightarrow \forall x\in D,\ \int\frac{\varphi(x)}{x^2} > 0$. Supposons maintenant que $\lambda \leqslant 0$, on sait que $f$ possède une limite finie et on a donc~:
    \begin{displaymath}
      \lim_{x\to +\infty} f(x) \leqslant 0 \quad\because\forall x\in D,\ \int\frac{\varphi(x)}{x^2} > 0
    \end{displaymath}
    Ce qui est absurde puisque $\lim_{x\to +\infty} f(x) = e > 0$, on en conclut que $\lambda > 0$, et donc par la règle des signes~: $\forall x\in D,\ f(x) > 0$. De même,
    \begin{displaymath}
      \forall x\in D,\ f'(x) = \frac{\varphi(x)}{x^2}f(x) > 0
    \end{displaymath}
    On peut maintenant construire le tableau de variations de $f$ sur $D$ (figure \ref{varf}) à l'aide de toutes les informations obtenues précédemment. La courbe obtenue (figure \ref{C}) est cohérente avec les résultats démontrés.


    \begin{figure}[h]
     \begin{center}
       \begin{tikzpicture}
         \tkzTabInit{ $x$ / 1, $f'(x)$ / 1, $f$ / 1.5}{$-\infty$, 0, $+\infty$}
         \tkzTabLine{,+, d, +,}
         \tkzTabVar{-/$\dfrac{1}{e}$, +D-/1, +/$e$,}
       \end{tikzpicture}
     \end{center}
     \caption{Tableau de variations de $f$.}
     \label{varf}
    \end{figure}

    \begin{figure}[ht]
      \begin{center}
        \begin{tikzpicture}[scale=0.96]
          \draw [thin, color=gray!25] (-8,-1) grid (8,4);
          \draw (-8,0) -- (8,0);
          \draw (0,-1) -- (0,4);
          \draw [domain=-8:8,samples=80] plot (\x,{cosh(\x)^(1/\x)} ) node [left, below] {$\mathscr{C}$};
          \draw [domain=-4:6,samples=2] plot (\x,{1+ \x/2}) node [left, below] {$\mathscr{T}_0$};
          \draw[->, >=stealth, thick] (0,0)--(1,0);
          \draw[->, >=stealth, thick] (0,0)--(0,1);
          \draw[dashed] (8, e)--(0, e) node[left]{$e$};
          \draw[dashed] (-8, 1/e)--(0, 1/e) node[right]{$1/e$};
          \node[below left] at (0,0){O};
        \end{tikzpicture}
      \end{center}
      \caption{Représentation de $\mathscr{C}$ et $\mathscr{T}_0$}
      \label{C}
    \end{figure}
\end{document}
